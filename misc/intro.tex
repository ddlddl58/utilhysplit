
\introduction  %% \introduction[modified heading if necessary]

Atmospheric transport and dispersion models (ATDM) are often used to model concentrations
of resuspended materials such as volcanic ash and minearl dust. In some applications a forecast of when and where
high concentrations of such materials can be expected is desired (cite). In other applications, the aim
is to determine the source of observed concentrations of such materials (cite).
%Here we are concerned with predicting concentrations of volcanic ash, but the technique 
%may be applied to other problems with minor modifications. 

ATDM require inputs from a numerical weather prediction (NWP) model and
information about the initial position and amount of material. To model resuspension with a foward modeling setup, the ATDM
requires information about the location of the source regions and the mass of material lifted from each source
region as a function of time. 

Dust and ash are resuspended
by a transfer of momentum from the atmosphere to the deposit. The amount of material resuspended
depends on both properties of the atmosphere and properties of the deposit ~\citep{Kok12}. 
Very rarely is there detailed information about deposit properties such as grain size distribution, soil moisture and deposit depth.
Often, even the horizontal extent of the deposit is not clearly known. 
Furthermore, deposit properties change over time. 
The NWP model supplies information about the atmosphere, in particular, the friction velocity which characterizes
the momentum flux. However, the spatial and time resolution of the data available may be fairly course and
reuspension depends on very local conditions. The presence of non-erodible elements may also reduce emissions.

Although the prospect for forecasting ash and dust concentrations resulting from resuspension may seem bleak, models are often
fairly successful (citations). Here we delve into why and what factors are most important for successful modeling.

%Here we investigate the sensitivity of model predictions to the form of the mass flux relationship.
%Here we investigate a Bayesian inverse scheme for tuning the source term relationship using observational data. 
%The purpose of the
%scheme is not to unearth a correct or universal source term relationship,
%We  use observational data to identify
%effective source term relationships for the given location, time, and ATDM, NWP model combination. 

The ATDM model used is HYSPLIT. HYSPLIT is the operational model
at four volcanic ash advisory centers which are responsible for issuing volcanic ash advisories (VAAs) in the event of
a resuspension episode which would affect aviation.
The Anchorage VAAC (information about how many warnings for resuspended ash they have produced in the last decade).

%The resuspension of dust 

%Modeling of resuspended materials such as dust and volcanic ash is

%Resuspension of volcanic ash is important.
%Iceland
%Katmai
%South America

%after any eruption
%Similarities and differences than dust.

%\section{Resuspension in Iceland}
%TEXT

