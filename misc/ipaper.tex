%% Copernicus Publications Manuscript Preparation Template for LaTeX Submissions
%% ---------------------------------
%% This template should be used for copernicus.cls
%% The class file and some style files are bundled in the Copernicus Latex Package, which can be downloaded from the different journal webpages.
%% For further assistance please contact Copernicus Publications at: production@copernicus.org
%% https://publications.copernicus.org/for_authors/manuscript_preparation.html


%% Please use the following documentclass and journal abbreviations for discussion papers and final revised papers.


%% 2-column papers and discussion papers
\documentclass[acp, manuscript]{copernicus}

%% Journal abbreviations (please use the same for discussion papers and final revised papers)

% Archives Animal Breeding (aab)
% Atmospheric Chemistry and Physics (acp)
% Advances in Geosciences (adgeo)
% Advances in Statistical Climatology, Meteorology and Oceanography (ascmo)
% Annales Geophysicae (angeo)
% ASTRA Proceedings (ap)
% Atmospheric Measurement Techniques (amt)
% Advances in Radio Science (ars)
% Advances in Science and Research (asr)
% Biogeosciences (bg)
% Climate of the Past (cp)
% Drinking Water Engineering and Science (dwes)
% Earth System Dynamics (esd)
% Earth Surface Dynamics (esurf)
% Earth System Science Data (essd)
% Fossil Record (fr)
% Geographica Helvetica (gh)
% Geoscientific Instrumentation, Methods and Data Systems (gi)
% Geoscientific Model Development (gmd)
% Hydrology and Earth System Sciences (hess)
% History of Geo- and Space Sciences (hgss)
% Journal of Micropalaeontology (jm)
% Journal of Sensors and Sensor Systems (jsss)
% Mechanical Sciences (ms)
% Natural Hazards and Earth System Sciences (nhess)
% Nonlinear Processes in Geophysics (npg)
% Ocean Science (os)
% Proceedings of the International Association of Hydrological Sciences (piahs)
% Primate Biology (pb)
% Scientific Drilling (sd)
% SOIL (soil)
% Solid Earth (se)
% The Cryosphere (tc)
% Web Ecology (we)
% Wind Energy Science (wes)


%% \usepackage commands included in the copernicus.cls:
%\usepackage[german, english]{babel}
%\usepackage{tabularx}
%\usepackage{cancel}
%\usepackage{multirow}
%\usepackage{supertabular}
%\usepackage{algorithmic}
%\usepackage{algorithm}
%\usepackage{amsthm}
%\usepackage{float}
%\usepackage{subfig}
%\usepackage{rotating}


\begin{document}

\title{Modeling the transport and dispersion of resuspended volcanic ash.}

% \Author[affil]{given_name}{surname}

\Author[]{Alice}{Crawford}
\Author[]{Christopher}{Loughner}
\Author[]{Ariel}{Stein}

\affil[]{NOAA ARL}
\affil[]{ADDRESS}

%% The [] brackets identify the author with the corresponding affiliation. 1, 2, 3, etc. should be inserted.

\runningtitle{Resuspended volcanic ash}

\runningauthor{TEXT}

\correspondence{NAME (EMAIL)}

\received{}
\pubdiscuss{} %% only important for two-stage journals
\revised{}
\accepted{}
\published{}

%% These dates will be inserted by Copernicus Publications during the typesetting process.

\firstpage{1}
\maketitle

\begin{abstract}
Resuspension of volcanic ash from a deposit in Iceland is modeled with HYSPLIT driven by meteorological
fields from the ECMWF ERA5 dataset as well as WRF. The source term for the resuspended ash is assumed to be
a function of friction velocity provided by numerical weather prediction model. 
Overall, timing of resuspension events can be predicted well.
\end{abstract}

\copyrightstatement{TEXT}

%----------------------------------------------
%\introduction  %% \introduction[modified heading if necessary]


\introduction  %% \introduction[modified heading if necessary]

Atmospheric transport and dispersion models (ATDM) are often used to model concentrations
of resuspended materials such as volcanic ash and minearl dust. In some applications a forecast of when and where
high concentrations of such materials can be expected is desired (cite). In other applications, the aim
is to determine the source of observed concentrations of such materials (cite).
%Here we are concerned with predicting concentrations of volcanic ash, but the technique 
%may be applied to other problems with minor modifications. 

ATDM require inputs from a numerical weather prediction (NWP) model and
information about the initial position and amount of material. To model resuspension with a foward modeling setup, the ATDM
requires information about the location of the source regions and the mass of material lifted from each source
region as a function of time. 

Dust and ash are resuspended
by a transfer of momentum from the atmosphere to the deposit. The amount of material resuspended
depends on both properties of the atmosphere and properties of the deposit ~\citep{Kok12}. 
Very rarely is there detailed information about deposit properties such as grain size distribution, soil moisture and deposit depth.
Often, even the horizontal extent of the deposit is not clearly known. 
Furthermore, deposit properties change over time. 
The NWP model supplies information about the atmosphere, in particular, the friction velocity which characterizes
the momentum flux. However, the spatial and time resolution of the data available may be fairly course and
reuspension depends on very local conditions. The presence of non-erodible elements may also reduce emissions.

Although the prospect for forecasting ash and dust concentrations resulting from resuspension may seem bleak, models are often
fairly successful (citations). Here we delve into why and what factors are most important for successful modeling.

%Here we investigate the sensitivity of model predictions to the form of the mass flux relationship.
%Here we investigate a Bayesian inverse scheme for tuning the source term relationship using observational data. 
%The purpose of the
%scheme is not to unearth a correct or universal source term relationship,
%We  use observational data to identify
%effective source term relationships for the given location, time, and ATDM, NWP model combination. 

The ATDM model used is HYSPLIT. HYSPLIT is the operational model
at four volcanic ash advisory centers which are responsible for issuing volcanic ash advisories (VAAs) in the event of
a resuspension episode which would affect aviation.
The Anchorage VAAC (information about how many warnings for resuspended ash they have produced in the last decade).

%The resuspension of dust 

%Modeling of resuspended materials such as dust and volcanic ash is

%Resuspension of volcanic ash is important.
%Iceland
%Katmai
%South America

%after any eruption
%Similarities and differences than dust.

%\section{Resuspension in Iceland}
%TEXT



%----------------------------------------------
\section{Measurements}
\label{sec:measurements}


The location of measurement sites is shown in Figure~\ref{fig:sources}.
Two urban stations, Grensasvegur and Hvaleyrarholt lie a couple hundred kilometers to the west and slightly north of the
source region. 
Hvolsvollur is located on the east edge of the source region. Heimaland is located within the southeast region of the source region.
Vik is located on the south edge of the source region.
Measurements of PM$_{10}$ at all the measurement sites is shown in Figure~\ref{fig:obs}. 
Letters are used to identify resuspension events in which elevated concentrations were observed at one or more of the sites. 
Five main resuspension events, A, B, C, D, E, were identified.
Event A resulted in the highest concentrations (up to 2000 $\mu \mathrm{g} \; \mathrm{m}^{-3}$) at Hvolsvollur and elevated concentrations of
a few hundred \ugm at the other sites. 

Measurements are the same as though used in ~\cite{Leadbetter12}.
Measurements from PM$_{10}$ monitors at  4 stations for dates May 20, 2010 through June 30, 2010.  
Two urban stations, Grensasvegur and Hvaleyrarholt. Grensasvergur is located near a busy road.
The Hvolsvollur and Heimaland stations are in rural areas.

%Measurements at one station for dates September 2010 through February 2011. These measurements are from an optical particle counter, OPC.
%The OPC detects particles in the range of 0.25 to 32 $\mu\mathrm{m}$. 





%----------------------------------------------
\section{Method}
In this section, we describe the Lagrangian transport and dispersion  model, HYSPLIT, the meteorological data sets used as input into HYSPLIT.


\subsection{NWP model}
Data from the European Center for Medium-Range Weather Forecasts (ECMWF) ERA5 global atmospheric reanalysis ~\citep{era5} was used
as input into HYSPLIT.
The data set has $0.3^o$ latitude longitude resolution and analyses every hour. We used the output on pressure levels.

We also use data from the WRF model (more info here) 27km, 9km, 3km

%However, if $z_o$ is calculated from 10m wind speeds and friction velocity using the law of the wall, we find values of
%$z_o \approx 0.1 \mathrm{cm}$ for points over water and $z_o \approx 4.7\mathrm{cm}$ for points over land are common.

Figure~\ref{fig:udist}(a) shows the joint histograms of friction velocity and 10m wind speed. 
%The two linear relationship observed correspond to source points over land and source points over water.
The solid lines show the law of the wall relationship with $z_o=0.068, 4.7, 10$, and $50 \mathrm{cm}$.  
They are not fits but simply shown for comparison. For the same wind speed, source points over water exibit lower
friction velocities, which corresponds to a lower roughness length. However, wind speeds and thus also friction velocities
tended to be higher over water.  
A forecast surface roughness is available with the ERA5 dataset (short-name flsr and paramID 244). 
It is zero for the source points over the ocean and ranges from zero to almost 90cm for
the source points over the land. The values for each source point do not vary more than a few cm for each source point over time.
The roughness lengths from the NWP model fields do not appear to correspond to the roughness length estimated using the law of the wall, particularly for the source points over land.

Figure~\ref{fig:udist}(b) and (c) shows the joint histogram of friction velocity and 10m wind speed for WRF 27km and 9km respectively. 
The solid lines are the same as shown in (a). 
One significant different between the WRF joint distributions and the ERA5 is that the highest friction velocities are
not associated with the highest wind speeds, instead they are associated with higher roughness lengths. 

Figure~\ref{fig:ustarA} shows friction velocities as a function of time for all source points for the ERA5 and WRF runs. 
Source points for ERA5 which lie over ocean are plotted separately from those which lie over land.
The letters indicate the time of
the five main resuspension events that were identified in Section~\ref{sec:measurements}.
A horizontal line is drawn at $u_*=30\mathrm{cm}\;\mathrm{s}^{-1}$ as threshold friction velocities are upward of this value.
The ERA5 and WRF runs show similar temporal patterns in values of friction velocities. Elevated $u_*$ is observed between A and B as well
as concurrent with C,D,E and F. WRF friction velocities are higher in general


%between the WRF 27km joint distribution and the ERA5 joint distribution is that top wind speeds are lower, but friction velocities are higher. The wind speeds may be lower, because we do not include 




%NWP model
%Modeling transport
%Modeling Emissions

%\section{Method}

\subsection{Modeling Transport}
HYSPLIT is a Lagrangian transport and dispersion model developed by the National Oceanic and Atmospheric Administration's Air Resources Laboratory (NOAA ARL).  
The model is used operationally at the  Washington, Anchorage, Darwin and Wellington volcanic ash advisory centers (VAACs) for modeling the transport and dispersion of volcanic ash{\color{blue}.} 
It is also used to provide 
forecasts of smoke from wild fires ~\citep{Rolph09, Stein09}, forecasts of windblown dust ~\citep{Draxler10}, and atmospheric dispersion products for chemical and nuclear accidents~\citep{Draxler12}. The model details and history are described in 
~\cite{Draxler97, Draxler98, bams}. 
%Here the model is used in particle (rather than puff) mode. The model is configured to calculate wet and dry deposition. 
%Column mass loadings are calculated over a 0.05$^o$ latitude by 0.1$^o$ longitude grid.  At the latitudes of interest, this corresponds to $\sim$ 6km $\times$ 6km grid.

HYSPLIT model runs were performed which release one unit of mass for each particle size from every source location every hour from May 20 to June 30, 2010.

Particle diameters of 1, 5, 10, and 20 $\mu\mathrm{m}$ were modeled. All particles were given a shape factor of 1.
The 1,5 and 5$\mu\mathrm{m}$ particles were given a density $2.5\mathrm{g}\;\mathrm{cc}^{-1}$ while the 20$\mu\mathrm{m}$ particles had a density of 
$2.2\mathrm{g}\;\mathrm{cc}^{-1}$. In HYSPLIT, the particle diameter, density and shape are used
to calculate a settling velocity. The ~\cite{Ganser93, dare} formulation was used to calculate settling velocity.
The output concentration grid for each run has resolution 50 meters in he vertical. Horizontal resolution for a run set was  either $0.2 \times 0.2$ or $0.05 \times 0.05$ degrees.
Table~\ref{tab:runs} associates a run set identifier with pertinent model inputs.
Model inputs not specified in the table remained the same for all run sets.


 \begin{table}
 \caption{Summary of sets of HYSPLIT runs.}
 \centering
 \begin{tabular}{l c c c}
 \hline
  Run Set  & NWP model & Concentration  & Number release points  \\
   & NWP model & Grid Resoluton & per source \\
 \hline
   E     & ERA5    & $0.2^o$  & 1  \\
   W27A  & WRF 27km& $0.2^o$  & 1  \\
   W27B  & WRF 27km& $0.05^o$ & 81 \\
   W9    & WRF 9km & $0.2^o$  & 1  \\
   W3    & WRF 3km & $0.05^o$ & 1  \\
 \hline
 %\multicolumn{2}{l}{$^{a}$Footnote text here.}
 \label{tab:runs}
 \end{tabular}
 \end{table}

\begin{table}
 \caption{Postprocessing}
 \centering
 \begin{tabular}{l c c c}
 \hline
  Identifer & Emission & Source set  & particle size  \\
 \hline
r1s4psd & WE & s4 & PSD \\ 
r1s4p5 & WE & s4 & 5um  \\

 \hline
 %\multicolumn{2}{l}{$^{a}$Footnote text here.}
 \label{tab:results}
 \end{tabular}
 \end{table}


Additionally a database of friction velocity, precipitation  and 10m wind speed at each source point at each hour was created from the NWP outputs.

Concentrations for each measurement station are extracted from each of the HYSPLIT  output files in unit mass per meter cubed. 
%The unit mass is converted
%to $\mu\mathrm{g}\mathrm{m}^-3$ using an expression which relates the flux of mass at the source to the precipitation and friction velocity. 
A transfer coefficient matrix, {\bf T}, is constructed in  
which the  values, $T_{ij}$,  are modeled concentrations (in unit mass per volume) for which each source, $i$,  contributes to each measurement, $j$.
The source is specified by the particle size, location of release, and time of release. 
The matrix can be multiplied by the emissions vector, \vec{E}, to obtain a modeled measurement vector, \vec{M},  with values for the forecast concentrations.

$$M_j = \sum_{i} E_i T_{ij} $$

Figure~\ref{fig:sources} shows the center of the source locations. These were chosen to coincide with the center of the NWP grid cells. For most runs, all particles
were released from the center point. For one run (runid 11) which was driven by the 27km WRF, particles were released at 3km intervals throughout the source area. Thus particle
release locations would coincide with particle release locations for the run driven by the 3km WRF.

One advantage of this setup is that once all the model runs are completed, different formulations for determining the emissions can
be applied relatively quickly without running HYSPLIT again. 
Although many HYSPLIT runs are needed, the individual runs are short and they can be run in parallel. However, computational time does
increase significantly at higher resolution. For the 3km WRF run, there are 81 source points for each single source in the 27km WRF run.
The transfer coefficient matrix is thus significantly larger as well.

Another advantage is that the effect of the modeled transport on the modeled concentrations can be examined separately from
the effect of the modeled emissions.   

%Here we optimze \vec{E} by matching \vec{M} with observations.  


\subsection{Modeling Emissions}

Dust emission schemes are numerous but generally 

An emissions vector \vec{E} is constructed by estimating the mass flux, $\Phi$, of material from each source.
Emissions 

\begin{equation}
\Phi(u_*, u_{*t},x) =\begin{cases}
G(x)u_*^N \;\;\; u_{*} \geq u_{*t} \\
0 \;\;\;\;\;\;  u_* < u_{*t} 
\end{cases}
\label{eq:massflux}
\end{equation}

A widely used empirical relationship used for the mass flux of dust developed in ~\citep{westphal}  
uses $N=4$ and $G=1\times10^{-5}$ is a constant. Other functional forms which have been suggested include the f

\begin{equation}
\Phi(u_*, u_{*t},x) =\begin{cases}
G(x)u_*^N (1-\frac{u_{*t}^{N-1}}{u_{*}^{N-1}})  \;\;\; u_{*} \geq u_{*t} \\
0 \;\;\;\;\;\;  u_* < u_{*t} 
\end{cases}
\label{eq:shao}
\end{equation}

%\begin{equation}
%\Phi(u_*, u_{*t}) =\begin{cases}
%1\times10^{-5}u_*^4 \;\;\; u_{*} \geq u_{*t} \\
%0 \;\;\;\;\;\;\;\;\;\;  u_* < u_{*t} 
%\end{cases}
%\label{eq:westphal}
%\end{equation}

$u_*$ is the friction velocity in $\mathrm{m}\;\mathrm{s}^{-1}$ and $u_{*t}$ is a threshold friction velocity.
The units of the empirical constant, $1\times10^{-5}$ are such that the vertical flux, $\Phi$ is in units of kg~m$^{-2}$~s$^{-1}$.

%Although there are more sophisticated relationships which take into account ......

To model resuspension of volcanic ash ~\cite{Leadbetter12} uses a source strength proportional to
$(u_{*}-u_{*t})^{3}$ and a threshold friction velocity $u_{*t} = 40 \mathrm{cm} \; \mathrm{s}^{-1}$. 
They find that using $u_{*t} = 50 \mathrm{cm} \; \mathrm{s}^{-1}$ resulted in missed or shortened resuspension events. 
They do not investigate the sensitivity of their results to changes in
the source strength relationship.

~\cite{Folch14} investigates the use of several relationships to model resuspension of volcanic ash in South America.
He uses the simple relationship in ~\cite{westphal} as well as relationships developed for dust in ~\cite{Marticorena97} and
~\cite{Shao93} and found that Equation~\ref{eq:westphal} resulted in the best model predictions, although the other
relationships performed adequately. All relationships needed to be scaled to reproduce the magnitude of resuspension events.

The ~\cite{Marticorena97} relationship is of the same form as equation~\ref{eq:shao} with N=3 and a  threshold friction velocity
that is dependant on particle size. 

~cite{Darmenova09} discusses the problems associated with using friction velocity from an NWP model (WRF with a resolution of in their case) directly in a dust emission scheme and suggests some remedies. 
A drag partition correction may be applied to account for the presence of non-erodible elements. In short, the drag partition correction partitions the momentum represented by the friction velocity into momentum which is transferred to non-erodible elements and momentum which is transferred to the bare erodible surface. 
~\cite{Darmenova} applies the partition correction only to the threshold friction velocity.
Friction velocity provided by the NWP model represents momentum transferred to the surface which generally contains non-erodible elements 

%$$E = K u_*(U_*^2-u_{*t}^2(d)) $$
%where the threshold friction velocity $u_{*t}$ is a function of particle diameter, $d$.

%\subsection{Testing sources}
%
%When modeling dust or ash resuspension, the modeler often has to contend with an uncertain source area. 
%In this case, the area around the volcano is assumed to have deposits of ash (citation). However, some areas may 
%Also, our grid of source locations is relatively course. Some grid squares contain both water and land.
%Here we apply a simple scheme to identify sources which should not be used.

%First concentrations at each measurement station due to each latitude longitude source are calculated and compared to
%measurements at the stations. The Pearson correlation coefficient is calculated for each latitude longitude source.
%Sources which produce concentrations with a high negative correlation with observations are removed from consideration.

\subsection{Particle Size Distribution}

While the particle size distribution of the deposit is expected to affect the magnitude of emissions, there is evidence that
both for dust and volcanic ash, it does not greatly affect the particle size distribution of the resuspended material (~\citep{Mahowald14,DRI}).
~\cite{Kok11, Kok11b, Mahowald14} present evidence that the PSD of resuspended dust particles less than $5\mu\mathrm{m}$ is independent of the PSD of the dust deposit and $u_{*t}$.
%There is no particular reason to believe that ash will behave similary to mineral dust in this respect. The brittle fragmentation theory proposed by ~\cite{Kok11} 

~\cite{DRI} conducted laboratory measurements of resuspension of ash collected from two different ash deposits. The PSDs of the bulk ash were quite different 
as were the measured emission relationsips. However, the PSDs of the resuspended ash was very similar. This suggests that, like dust, the PSD of the resuspended material may be treated as independant of the PSD of the parent deposit. 
~\cite{Folch14} found that emission schemes in which threshold friction velocity (and thus emissions) was dependent on particle size did not perform better than
a simple emission scheme which did

%Thus it is reasonable to model emissions using a relationship which predicts total mass flux of a certain particle size range (for instance PM10) 
%and possibly as a function of deposit properties including deposit PSD and then distribute the mass according to an empirically determined resuspended particle size distribution.

When modeling transport, the particle size, shape and density of the modeled material is used to calculate a gravitational settling velocity as well as the deposition velocity.
Computational particles with the same same settling velocity will behave identically in the model. 
The fall velocities for these particle sizes calculated using the formulation in ~\cite{Ganser} are 
$7.6\times 10^{-7}, 7.6\times10^{-5}, 1.9\times10^{-3}, 7.6\times10^{-3}, 3.0\times10^{-2} \mathrm{m}\mathrm{s}^{-1}$ for the 0.1, 1,5,10 and 20 $\mu\mathrm{m}$ particles respectively. 
Each particle size bin represents particles with similar settling velocities. 
The settling velocity affects modeled transport in
two main ways. Gravitational settling combined with wind
shear can cause different size particles to be transported in different directions and at different speeds. 
Deposition will decrease the amount of material in the transported plume, with larger particles remaining closer to the ground and losing more mass.

In our setup, we can compare transport of the different particle sizes by applying a constant emissions vector and comparing concentrations (in arbitrary units) due to each
particle size. 

\subsection{Transport differences}

Differences in modeled concentrations can be due to either differences in modeled emissions or modeled transport. 
In Run 10 particles were released only from the center of each source point, while in Run 11 particles were released 

\subsection{Drag partition correction}

Modeling emissions using equation~\ref{eq:westphal} and modeling transport with HYSPLIT driven by ERA5 produced quite good predictions. These are shown in Figure~\ref{fig:run1}.
For comparision, prediction using constant emissions are shown as well. Clearly, modeling emissions correctly is important.

However, when HYSPLIT was driven by the 27km WRF, model predictions were overestimated significantly. 
Figure~\ref{fig:wind_ustar} shows time series of 10 meter winds and friction velocities for all the source points for both the ERA5
and the 27km WRF. Friction velocities output by WRF are higher and emissions are quite sensitive to these values. 

The overestimation is even more extreme for the 9km WRF  and 3km WRF.
The 9km WRF runs output friction velocities up to 160 cm/s, more than twice the value of the highest .

The friction velocity provided by the NWP model represents the total shear stress on the surface and generally areas with higher aerodynamic roughness will 



~\cite{Darmenova} suggests calculating $u_*$ from ten meter wind speeds using Equation~\ref{eq:lawofwall} and an aeolian roughness length which may be much smaller than aerodynamic roughness length from the NWP model.  Doing so is similar to applying the 
drag partition correction described in~\citep{Marticorena97,MacKinnon04,Darmenova09}. 
In a neutral PBL where the velocity profile is logarithmic, the relcalulated friction velocity, $u_*'$, will be related to the model friction velocity $u_*$ in the following way

$$\frac{u_*'}{u_*} = 1 - \frac{\mathrm{ln}\frac{z_o}{z_{os}}}{\mathrm{ln}\frac{10\mathrm{m}}{z_{os}}} $$

%~\cite{Marticorena97} assumes an internal boundary layer, IBL, develops between roughness elements and there is a height, h, at which the logarithmic velocity profile of the IBL transitions to the logarithmic velocity profile of the atmospheric boundary layer. 

Following this procedure using  $z_0=0.001m$  resulted in model predictions which were comparable to those produced with the ERA5 data. 

Figure~\ref{fig:ustar27} plots friction velocity output by the model vs. friction velocity calculated from ten meter wind speeds. Figure~\ref{fig:run11} shows model predictions and observations at the five measurement sites.

However, using this procedure on the modeled emissions with the ERA5 data resulted in much worse predictions.

For the 9km WRF, using $z_0=0.0001m$ yielded  comparable results.

It is worth noting that in ~\cite{Darmenova09} and elsewhere, the drag partition correction is applied to the threshold friction velocity and the vertical mass flux is.


\subsection{Sensitivity Analysis}

Here we investigate the relationship between the emissions and the friction velocity, $u_*$, provided by the meteorological model.
To begin with, it is assumed that the mass flux is related to the friction velocity in the following manner.

\begin{equation}
\Phi(u_*, u_{*t}) =\begin{cases}
Au_*^N \;\;\; u_{*} \geq u_{*t} \\
0 \;\;\;\;\;\;  u_* < u_{*t} 
\end{cases}
\label{eq:massflux}
\end{equation}

Where A, N, and $u_{*t}$ are unknown. 
%Other unknowns not investigated here include the particle size distribution, source area extent, depenedence on precipitation.

We assume that $A$ can be any
positive real number. $N$ and $u_{*t}$ can take on any of the following values, in any combination.
$$ u_{*t} = 10,15,20,25,30,35,40,45,50 \; \mathrm{cm} \; \mathrm{s}^{-1}$$
$$ N = 1,2,3,4,5,6,7,8,9 $$

We also look at the following relationships

\begin{equation}
\Phi(u_*, u_{*t}) =\begin{cases}
A(u_*^3 - u_{*t}^3)  \;\;\; u_{*} \geq u_{*t} \\
0 \;\;\;\;\;\;  u_* < u_{*t} 
\end{cases}
\label{eq:leadbetter}
\end{equation}

%This same as next one.
%\begin{equation}
%\Phi(u_*, u_{*t}) =\begin{cases}
%Au_* (u_*^2 - u_{*t}^2)  \;\;\; u_{*} \geq u_{*t} \\
%0 \;\;\;\;\;\;  u_* < u_{*t} 
%\end{cases}
%\label{eq:marticorena}
%\end{equation}

\begin{equation}
\Phi(u_*, u_{*t}) =\begin{cases}
Au_*^3 (1-\frac{u_{*t}^2}{u_{*}^2})  \;\;\; u_{*} \geq u_{*t} \\
0 \;\;\;\;\;\;  u_* < u_{*t} 
\end{cases}
\label{eq:shao}
\end{equation}

\begin{equation}
\Phi(u_*, u_{*t}) =\begin{cases}
Au_*^4 (1-\frac{u_{*t}}{u_{*}})  \;\;\; u_{*} \geq u_{*t} \\
0 \;\;\;\;\;\;  u_* < u_{*t} 
\end{cases}
\label{eq:gillette}
\end{equation}

These functional forms have been suggested in ~\cite{Shao08}, ~\cite{Gillette}, ~\cite{Marticorena}, ~\cite{Leadbetter12}.

The constant $A$ is estimated by plotting observations vs. forecasts and performing a linear regression to find the slope.

????
To reduce bias, cumulative distribution function, CDF,  matching is performed. CDF matching can help account for
background concentrations.

Then the Pearson correlation coefficient, $r$,  and root mean square error, rmse, are calculated 



%subsection NWP model
%subsection Modeling Transport
%subsection Modeling Emissions
%subsection Particle Size Distribution
%subsection Transport differences
%subsection Drag Partition Correction
%subsection Sensitivity Analysis

%----------------------------------------------
%\section{Results}

\section{Results}

%-------------------------------------------
Modeling emissions using equation~\ref{eq:westphal} and modeling transport with HYSPLIT driven by ERA5 produced quite good predictions. These are shown in Figure~\ref{fig:run1}.
For comparision, prediction using constant emissions are shown as well. Clearly, modeling emissions correctly is important.

However, when HYSPLIT was driven by the 27km WRF, model predictions were overestimated significantly. 
Figure~\ref{fig:wind_ustar} shows time series of 10 meter winds and friction velocities for all the source points for both the ERA5
and the 27km WRF. Friction velocities output by WRF are higher and emissions are quite sensitive to these values. 

The overestimation is even more extreme for the 9km WRF  and 3km WRF.
The 9km WRF runs output friction velocities up to 160 cm/s, more than twice the value of the highest .

%-------------------------------------------

\subsection{Importance of transport}

Predicting high ash concentrations at locations farther away from the source may be easier than predicting high ash concentraions for more proximal locations. For instance, the :w
:q

It bears mentioning that
modeling transport correctly was more important for stations which were further away from the source, while modeling emissions
correctly was more important for stations close to the source.
Also, the differences between w27A and w27B were smaller at stations further away (Hval and Gren). 

The friction velocity provided by the NWP model represents the total shear stress on the surface and generally areas with higher aerodynamic roughness will 


Figure~\ref{fig:correlationsa} shows the correlation coefficient as a function of functional form, $N$, $u_{*t}$ for the period
of 5/25/2010 through 6/30/2010 using PM$_{10}$ measurements at the four stations shown in Figure~\ref{fig:locations}.
In the plots, $N=-1, -2, -3, -4$ corresponds to the use of Equations~\ref{eq:leadbetter}, ~\ref{eq:marticorena}, ~\ref{eq:shao}, ~\ref{gillette} respectively.

Figure~\ref{fig:rmse}

%Figure~\ref{fig:correlationsc} shows the correlation coefficient as a function of $N$ and $u_{*t}$ for the period
%of 10/01/2010 through 2/15/2011 using measurements from the OPC at Drangshilidardalur.

For the $PM_{10}$ meausurements, sources which experienced friction velocities greater than $45 \mathrm{cm} \; \mathrm{s}^{-1}$
did not place ash at any of the meausurement sites, thus placing the friction velocity above 50~cm~s$^{-1}$ resulted in zero
correlation.  
The more likely releationships tend to  pair higher values of $N$ with lower values of $u_{*t}$. This indicates that while
contributions from sources with low values of friction velocity are important, they must release much less mass than sources
which experience high values of $u_{*}$.

For values of $N$ lower than 4 or 5, values of 30 or 35 for $u_{*t}$ seem likely to result in higher correlations 
 which is consistent with what
has been seen in the literature ~\citep{Leadbetter12, Folch14}.

The most likely relationships found with the $PM_{10}$ measurements from the period of 5/25/201 to 6/30/2010 have
a high overlap with the most likely relationships found with the OPC counter from 10/1/2011 to 2/15/2011.
This demonstrates that assimilating measurements in this manner has potential for improving resuspended ash forecasts.

Modeled concentrations are compared to measurements at each of the stations in Figures~\ref{fig:concplota} through ~\ref{fig:concplotskogar}.
Three modeled concentrations which utilize different values of $N$ and $u_{*t}$ are shown in each plot.
For the PM$_{10}$ measurements, $N=8$ and $u_{*t}=25$ was the most likely, whie for the OPC measurements, $N=7$ and $u_{*t}=10$
was the most likely.  As can be seen, the correlations with the OPC measurements were much lower than the correlations with the PM10 meausurements.
Several high concentration events which were measured by the OPC were not captured by the model and the duration of modeled events tended to be
too short. It may be that some sources were missing for this. This was the only station located within the resuspended source area. It could
be the case that localized conditions not captured by the NWP model caused some resuspension events.

Figure~\ref{fig:concplota} shows concentrations as a function of time at the Heimaland station.
Figure~\ref{fig:concplotb} shows concentrations as a function of time at the Hvolsvollur station.
Figure~\ref{fig:concplotc} shows concentrations as a function of time at the Hvaleyrarholt station.
Figure~\ref{fig:concplotd} shows concentrations as a function of time at the Grensavegur station.
Figure~\ref{fig:concplotskogar} shows concentrations as a function of time at the Drangshilidardalur station.


\conclusions  %% \conclusions[modified heading if necessary]

The results agree with the invesgitation by ~\cite{Folch14} which predicted resuspended ash concentrations in South America using functional
relationships Equation~\ref{eq:shao}, ~\ref{eq:marticorena}, and ~\ref{eq:massflux} with $N=4$ and $u_{*t}=30\mathrm{cm}\;\mathrm{s}^{-1}$.
~\cite{Folch14} found that the simplest emission scheme 

%Transport and dispersion models aimed at forecasting resuspension of volcanic ash should consider using
%relatively low cutoff thresholds for friction velocity, of only 10 to 25 cm~s$^-1$ paired with a mass flux
%relationship which specifies a steep increase in mass flux with $u_*$. 

%Since deposit properties may change over time and from area to area, 
%a Baysian inference scheme such as demonstrated here
%can provide information on the most effective mass flux relationships to use for the area and model. 
%Here we show that data collected over a month period at four measurement stations improved forecasts for a subsequent 4 month time period
%at a different location.
(more work in the results section to show this is the case but expect it to be so as liklihoods overlapped significantly).

This scheme may also be useful for applying to dust storms.








%% The following commands are for the statements about the availability of data sets and/or software code corresponding to the manuscript.
%% It is strongly recommended to make use of these sections in case data sets and/or software code have been part of your research the article is based on.

%\codeavailability{TEXT} %% use this section when having only software code available


%\dataavailability{TEXT} %% use this section when having only data sets available


%\codedataavailability{TEXT} %% use this section when having data sets and software code available





%\appendix
%\section{}    %% Appendix A

%\subsection{}     %% Appendix A1, A2, etc.


%\noappendix       %% use this to mark the end of the appendix section

%% Regarding figures and tables in appendices, the following two options are possible depending on your general handling of figures and tables in the manuscript environment:

%% Option 1: If you sorted all figures and tables into the sections of the text, please also sort the appendix figures and appendix tables into the respective appendix sections.
%% They will be correctly named automatically.

%% Option 2: If you put all figures after the reference list, please insert appendix tables and figures after the normal tables and figures.
%% To rename them correctly to A1, A2, etc., please add the following commands in front of them:

%\appendixfigures  %% needs to be added in front of appendix figures

%\appendixtables   %% needs to be added in front of appendix tables

%% Please add \clearpage between each table and/or figure. Further guidelines on figures and tables can be found below.



%\authorcontribution{TEXT} %% optional section
%
%\competinginterests{TEXT} %% this section is mandatory even if you declare that no competing interests are present

%\disclaimer{TEXT} %% optional section

%\begin{acknowledgements}
%TEXT
%\end{acknowledgements}




%% REFERENCES

\bibliography{esp,resuspension2}{}
\bibliographystyle{agufull08}

%% The reference list is compiled as follows:

%\begin{thebibliography}{}

%\bibitem[AUTHOR(YEAR)]{LABEL}
%REFERENCE 1

%\bibitem[AUTHOR(YEAR)]{LABEL}
%REFERENCE 2

%\end{thebibliography}

%% Since the Copernicus LaTeX package includes the BibTeX style file copernicus.bst,
%% authors experienced with BibTeX only have to include the following two lines:
%%
%% \bibliographystyle{copernicus}
%% \bibliography{example.bib}
%%
%% URLs and DOIs can be entered in your BibTeX file as:
%%
%% URL = {http://www.xyz.org/~jones/idx_g.htm}
%% DOI = {10.5194/xyz}


%% LITERATURE CITATIONS
%%
%% command                        & example result
%% \citet{jones90}|               & Jones et al. (1990)
%% \citep{jones90}|               & (Jones et al., 1990)
%% \citep{jones90,jones93}|       & (Jones et al., 1990, 1993)
%% \citep[p.~32]{jones90}|        & (Jones et al., 1990, p.~32)
%% \citep[e.g.,][]{jones90}|      & (e.g., Jones et al., 1990)
%% \citep[e.g.,][p.~32]{jones90}| & (e.g., Jones et al., 1990, p.~32)
%% \citeauthor{jones90}|          & Jones et al.
%% \citeyear{jones90}|            & 1990



%% FIGURES

%% When figures and tables are placed at the end of the MS (article in one-column style), please add \clearpage
%% between bibliography and first table and/or figure as well as between each table and/or figure.

% ONE-COLUMN FIGURES
%plot_locations.py
\begin{figure}[t]
\includegraphics[width=15cm]{locations.png}
\caption{Location of measurements sites (green squares), Eyjafjalloj\"{o}kull (black triangle), and HYSPLIT sources (ERA5 red diamonds, WRF 27km Blue diamonds, WRF 9km cyan circles}
\label{fig:sources}
\end{figure}
%o
%plot_obs.py
\begin{figure}[p]
\includegraphics[width=10cm]{obs.pdf}
\caption{Measurements at each location. Letters are placed at the same time to indicate times at which elevated PM$_{10}$ levels
were measured at one or more locations.}
\label{fig:obs}
\end{figure}
%
%plot_zo.py
\begin{figure}[p]
\includegraphics[width=10cm]{uhist.pdf}
\caption{Joint histograms of friction velocity and ten meter wind speed for (a)ERA5, (b)WRF 27km, (c) WRF 9km, (d) WRF 3km,
with individual histograms located along the axis.
The solid lines show law of the wall relationships with surface roughness values of $z_o=$0.068cm (blue), 4.7cm(black), 10cm (red), and 50cm(green). The solid lines are the same in each graph. }
\label{fig:uhist}
\end{figure}
%
%plot_ustar.py
\begin{figure}[p]
\includegraphics[width=10cm]{ustar_ut.pdf}
\caption{Friction velocity and 10m wind speeds as a function of time for (a)ERA5 over land, (b)ERA5 over ocean,
(c)WRF 27km, (d) WRF 9 km. The horizontal line is drawn at 30~cm~s$^{-1}$. The capital letters are placed at the same times as
those in Figure~\ref{fig:obs}.}
\label{fig:ustar_time}
\end{figure}

%plot_westphal.py
\begin{figure}[p]
\includegraphics[width=16cm]{westphal.pdf}
%\caption{Forecast at each location. Scale and letter placement is the same as in Figure~\ref{fig:obs}. Panels on the
%left show forecast in $\mu\mathrm{g}\;\mathrm{m}^{-3}$ with emissions estimated using the Westphal relationship. Panels
%on the right show forecast in arbitrary units assuming constant emissions for $u_*>5 \mathrm{cm}\;\mathrm{s^{-1}$.}
\label{fig:westphal}
\end{figure}
%





%%f
%\begin{figure}[t]
%\includegraphics[width=16cm]{FILE NAME}
%\caption{TEXT}
%\end{figure}
%
%%% TWO-COLUMN FIGURES
%
%%f
%\begin{figure*}[t]
%\includegraphics[width=12cm]{FILE NAME}
%\caption{TEXT}
%\end{figure*}
%
%
%%% TABLES
%%%
%%% The different columns must be seperated with a & command and should
%%% end with \\ to identify the column brake.
%
%%% ONE-COLUMN TABLE
%
%%t
%\begin{table}[t]
%\caption{TEXT}
%\begin{tabular}{column = lcr}
%\tophline
%
%\middlehline
%
%\bottomhline
%\end{tabular}
%\belowtable{} % Table Footnotes
%\end{table}
%
%%% TWO-COLUMN TABLE
%
%%t
%\begin{table*}[t]
%\caption{TEXT}
%\begin{tabular}{column = lcr}
%\tophline
%
%\middlehline
%
%\bottomhline
%\end{tabular}
%\belowtable{} % Table Footnotes
%\end{table*}
%
%
%%% MATHEMATICAL EXPRESSIONS
%
%%% All papers typeset by Copernicus Publications follow the math typesetting regulations
%%% given by the IUPAC Green Book (IUPAC: Quantities, Units and Symbols in Physical Chemistry,
%%% 2nd Edn., Blackwell Science, available at: http://old.iupac.org/publications/books/gbook/green_book_2ed.pdf, 1993).
%%%
%%% Physical quantities/variables are typeset in italic font (t for time, T for Temperature)
%%% Indices which are not defined are typeset in italic font (x, y, z, a, b, c)
%%% Items/objects which are defined are typeset in roman font (Car A, Car B)
%%% Descriptions/specifications which are defined by itself are typeset in roman font (abs, rel, ref, tot, net, ice)
%%% Abbreviations from 2 letters are typeset in roman font (RH, LAI)
%%% Vectors are identified in bold italic font using \vec{x}
%%% Matrices are identified in bold roman font
%%% Multiplication signs are typeset using the LaTeX commands \times (for vector products, grids, and exponential notations) or \cdot
%%% The character * should not be applied as mutliplication sign
%
%
%%% EQUATIONS
%
%%% Single-row equation
%
%\begin{equation}
%
%\end{equation}
%
%%% Multiline equation
%
%\begin{align}
%& 3 + 5 = 8\\
%& 3 + 5 = 8\\
%& 3 + 5 = 8
%\end{align}
%
%
%%% MATRICES
%
%\begin{matrix}
%x & y & z\\
%x & y & z\\
%x & y & z\\
%\end{matrix}
%
%
%%% ALGORITHM
%
%\begin{algorithm}
%\caption{�}
%\label{a1}
%\begin{algorithmic}
%�
%\end{algorithmic}
%\end{algorithm}
%
%
%%% CHEMICAL FORMULAS AND REACTIONS
%
%%% For formulas embedded in the text, please use \chem{}
%
%%% The reaction environment creates labels including the letter R, i.e. (R1), (R2), etc.
%
%\begin{reaction}
%%% \rightarrow should be used for normal (one-way) chemical reactions
%%% \rightleftharpoons should be used for equilibria
%%% \leftrightarrow should be used for resonance structures
%\end{reaction}
%
%
%%% PHYSICAL UNITS
%%%
%%% Please use \unit{} and apply the exponential notation


\end{document}
