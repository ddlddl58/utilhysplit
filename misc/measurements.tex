

The location of measurement sites is shown in Figure~\ref{fig:sources}.
Two urban stations, Grensasvegur and Hvaleyrarholt lie a couple hundred kilometers to the west and slightly north of the
source region. 
Hvolsvollur is located on the east edge of the source region. Heimaland is located within the southeast region of the source region.
Vik is located on the south edge of the source region.
Measurements of PM$_{10}$ at all the measurement sites is shown in Figure~\ref{fig:obs}. 
Letters are used to identify resuspension events in which elevated concentrations were observed at one or more of the sites. 
Five main resuspension events, A, B, C, D, E, were identified.
Event A resulted in the highest concentrations (up to 2000 $\mu \mathrm{g} \; \mathrm{m}^{-3}$) at Hvolsvollur and elevated concentrations of
a few hundred \ugm at the other sites. 

Measurements are the same as though used in ~\cite{Leadbetter12}.
Measurements from PM$_{10}$ monitors at  4 stations for dates May 20, 2010 through June 30, 2010.  
Two urban stations, Grensasvegur and Hvaleyrarholt. Grensasvergur is located near a busy road.
The Hvolsvollur and Heimaland stations are in rural areas.

%Measurements at one station for dates September 2010 through February 2011. These measurements are from an optical particle counter, OPC.
%The OPC detects particles in the range of 0.25 to 32 $\mu\mathrm{m}$. 



