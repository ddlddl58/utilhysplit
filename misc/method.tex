
%NWP model
%Modeling transport
%Modeling Emissions

%\section{Method}

\subsection{Modeling Transport}
HYSPLIT is a Lagrangian transport and dispersion model developed by the National Oceanic and Atmospheric Administration's Air Resources Laboratory (NOAA ARL).  
The model is used operationally at the  Washington, Anchorage, Darwin and Wellington volcanic ash advisory centers (VAACs) for modeling the transport and dispersion of volcanic ash{\color{blue}.} 
It is also used to provide 
forecasts of smoke from wild fires ~\citep{Rolph09, Stein09}, forecasts of windblown dust ~\citep{Draxler10}, and atmospheric dispersion products for chemical and nuclear accidents~\citep{Draxler12}. The model details and history are described in 
~\cite{Draxler97, Draxler98, bams}. 
%Here the model is used in particle (rather than puff) mode. The model is configured to calculate wet and dry deposition. 
%Column mass loadings are calculated over a 0.05$^o$ latitude by 0.1$^o$ longitude grid.  At the latitudes of interest, this corresponds to $\sim$ 6km $\times$ 6km grid.

HYSPLIT model runs were performed which release one unit of mass for each particle size from every source location every hour from May 20 to June 30, 2010.

Particle diameters of 1, 5, 10, and 20 $\mu\mathrm{m}$ were modeled. All particles were given a shape factor of 1.
The 1,5 and 5$\mu\mathrm{m}$ particles were given a density $2.5\mathrm{g}\;\mathrm{cc}^{-1}$ while the 20$\mu\mathrm{m}$ particles had a density of 
$2.2\mathrm{g}\;\mathrm{cc}^{-1}$. In HYSPLIT, the particle diameter, density and shape are used
to calculate a settling velocity. The ~\cite{Ganser93, dare} formulation was used to calculate settling velocity.
The output concentration grid for each run has resolution 50 meters in he vertical. Horizontal resolution for a run set was  either $0.2 \times 0.2$ or $0.05 \times 0.05$ degrees.
Table~\ref{tab:runs} associates a run set identifier with pertinent model inputs.
Model inputs not specified in the table remained the same for all run sets.


 \begin{table}
 \caption{Summary of sets of HYSPLIT runs.}
 \centering
 \begin{tabular}{l c c c}
 \hline
  Run Set  & NWP model & Concentration  & Number release points  \\
   & NWP model & Grid Resoluton & per source \\
 \hline
   E     & ERA5    & $0.2^o$  & 1  \\
   W27A  & WRF 27km& $0.2^o$  & 1  \\
   W27B  & WRF 27km& $0.05^o$ & 81 \\
   W9    & WRF 9km & $0.2^o$  & 1  \\
   W3    & WRF 3km & $0.05^o$ & 1  \\
 \hline
 %\multicolumn{2}{l}{$^{a}$Footnote text here.}
 \label{tab:runs}
 \end{tabular}
 \end{table}

\begin{table}
 \caption{Postprocessing}
 \centering
 \begin{tabular}{l c c c}
 \hline
  Identifer & Emission & Source set  & particle size  \\
 \hline
r1s4psd & WE & s4 & PSD \\ 
r1s4p5 & WE & s4 & 5um  \\

 \hline
 %\multicolumn{2}{l}{$^{a}$Footnote text here.}
 \label{tab:results}
 \end{tabular}
 \end{table}


Additionally a database of friction velocity, precipitation  and 10m wind speed at each source point at each hour was created from the NWP outputs.

Concentrations for each measurement station are extracted from each of the HYSPLIT  output files in unit mass per meter cubed. 
%The unit mass is converted
%to $\mu\mathrm{g}\mathrm{m}^-3$ using an expression which relates the flux of mass at the source to the precipitation and friction velocity. 
A transfer coefficient matrix, {\bf T}, is constructed in  
which the  values, $T_{ij}$,  are modeled concentrations (in unit mass per volume) for which each source, $i$,  contributes to each measurement, $j$.
The source is specified by the particle size, location of release, and time of release. 
The matrix can be multiplied by the emissions vector, \vec{E}, to obtain a modeled measurement vector, \vec{M},  with values for the forecast concentrations.

$$M_j = \sum_{i} E_i T_{ij} $$

Figure~\ref{fig:sources} shows the center of the source locations. These were chosen to coincide with the center of the NWP grid cells. For most runs, all particles
were released from the center point. For one run (runid 11) which was driven by the 27km WRF, particles were released at 3km intervals throughout the source area. Thus particle
release locations would coincide with particle release locations for the run driven by the 3km WRF.

One advantage of this setup is that once all the model runs are completed, different formulations for determining the emissions can
be applied relatively quickly without running HYSPLIT again. 
Although many HYSPLIT runs are needed, the individual runs are short and they can be run in parallel. However, computational time does
increase significantly at higher resolution. For the 3km WRF run, there are 81 source points for each single source in the 27km WRF run.
The transfer coefficient matrix is thus significantly larger as well.

Another advantage is that the effect of the modeled transport on the modeled concentrations can be examined separately from
the effect of the modeled emissions.   

%Here we optimze \vec{E} by matching \vec{M} with observations.  


\subsection{Modeling Emissions}

Dust emission schemes are numerous but generally 

An emissions vector \vec{E} is constructed by estimating the mass flux, $\Phi$, of material from each source.
Emissions 

\begin{equation}
\Phi(u_*, u_{*t},x) =\begin{cases}
G(x)u_*^N \;\;\; u_{*} \geq u_{*t} \\
0 \;\;\;\;\;\;  u_* < u_{*t} 
\end{cases}
\label{eq:massflux}
\end{equation}

A widely used empirical relationship used for the mass flux of dust developed in ~\citep{westphal}  
uses $N=4$ and $G=1\times10^{-5}$ is a constant. Other functional forms which have been suggested include the f

\begin{equation}
\Phi(u_*, u_{*t},x) =\begin{cases}
G(x)u_*^N (1-\frac{u_{*t}^{N-1}}{u_{*}^{N-1}})  \;\;\; u_{*} \geq u_{*t} \\
0 \;\;\;\;\;\;  u_* < u_{*t} 
\end{cases}
\label{eq:shao}
\end{equation}

%\begin{equation}
%\Phi(u_*, u_{*t}) =\begin{cases}
%1\times10^{-5}u_*^4 \;\;\; u_{*} \geq u_{*t} \\
%0 \;\;\;\;\;\;\;\;\;\;  u_* < u_{*t} 
%\end{cases}
%\label{eq:westphal}
%\end{equation}

$u_*$ is the friction velocity in $\mathrm{m}\;\mathrm{s}^{-1}$ and $u_{*t}$ is a threshold friction velocity.
The units of the empirical constant, $1\times10^{-5}$ are such that the vertical flux, $\Phi$ is in units of kg~m$^{-2}$~s$^{-1}$.

%Although there are more sophisticated relationships which take into account ......

To model resuspension of volcanic ash ~\cite{Leadbetter12} uses a source strength proportional to
$(u_{*}-u_{*t})^{3}$ and a threshold friction velocity $u_{*t} = 40 \mathrm{cm} \; \mathrm{s}^{-1}$. 
They find that using $u_{*t} = 50 \mathrm{cm} \; \mathrm{s}^{-1}$ resulted in missed or shortened resuspension events. 
They do not investigate the sensitivity of their results to changes in
the source strength relationship.

~\cite{Folch14} investigates the use of several relationships to model resuspension of volcanic ash in South America.
He uses the simple relationship in ~\cite{westphal} as well as relationships developed for dust in ~\cite{Marticorena97} and
~\cite{Shao93} and found that Equation~\ref{eq:westphal} resulted in the best model predictions, although the other
relationships performed adequately. All relationships needed to be scaled to reproduce the magnitude of resuspension events.

The ~\cite{Marticorena97} relationship is of the same form as equation~\ref{eq:shao} with N=3 and a  threshold friction velocity
that is dependant on particle size. 

~cite{Darmenova09} discusses the problems associated with using friction velocity from an NWP model (WRF with a resolution of in their case) directly in a dust emission scheme and suggests some remedies. 
A drag partition correction may be applied to account for the presence of non-erodible elements. In short, the drag partition correction partitions the momentum represented by the friction velocity into momentum which is transferred to non-erodible elements and momentum which is transferred to the bare erodible surface. 
~\cite{Darmenova} applies the partition correction only to the threshold friction velocity.
Friction velocity provided by the NWP model represents momentum transferred to the surface which generally contains non-erodible elements 

%$$E = K u_*(U_*^2-u_{*t}^2(d)) $$
%where the threshold friction velocity $u_{*t}$ is a function of particle diameter, $d$.

%\subsection{Testing sources}
%
%When modeling dust or ash resuspension, the modeler often has to contend with an uncertain source area. 
%In this case, the area around the volcano is assumed to have deposits of ash (citation). However, some areas may 
%Also, our grid of source locations is relatively course. Some grid squares contain both water and land.
%Here we apply a simple scheme to identify sources which should not be used.

%First concentrations at each measurement station due to each latitude longitude source are calculated and compared to
%measurements at the stations. The Pearson correlation coefficient is calculated for each latitude longitude source.
%Sources which produce concentrations with a high negative correlation with observations are removed from consideration.

\subsection{Particle Size Distribution}

While the particle size distribution of the deposit is expected to affect the magnitude of emissions, there is evidence that
both for dust and volcanic ash, it does not greatly affect the particle size distribution of the resuspended material (~\citep{Mahowald14,DRI}).
~\cite{Kok11, Kok11b, Mahowald14} present evidence that the PSD of resuspended dust particles less than $5\mu\mathrm{m}$ is independent of the PSD of the dust deposit and $u_{*t}$.
%There is no particular reason to believe that ash will behave similary to mineral dust in this respect. The brittle fragmentation theory proposed by ~\cite{Kok11} 

~\cite{DRI} conducted laboratory measurements of resuspension of ash collected from two different ash deposits. The PSDs of the bulk ash were quite different 
as were the measured emission relationsips. However, the PSDs of the resuspended ash was very similar. This suggests that, like dust, the PSD of the resuspended material may be treated as independant of the PSD of the parent deposit. 
~\cite{Folch14} found that emission schemes in which threshold friction velocity (and thus emissions) was dependent on particle size did not perform better than
a simple emission scheme which did

%Thus it is reasonable to model emissions using a relationship which predicts total mass flux of a certain particle size range (for instance PM10) 
%and possibly as a function of deposit properties including deposit PSD and then distribute the mass according to an empirically determined resuspended particle size distribution.

When modeling transport, the particle size, shape and density of the modeled material is used to calculate a gravitational settling velocity as well as the deposition velocity.
Computational particles with the same same settling velocity will behave identically in the model. 
The fall velocities for these particle sizes calculated using the formulation in ~\cite{Ganser} are 
$7.6\times 10^{-7}, 7.6\times10^{-5}, 1.9\times10^{-3}, 7.6\times10^{-3}, 3.0\times10^{-2} \mathrm{m}\mathrm{s}^{-1}$ for the 0.1, 1,5,10 and 20 $\mu\mathrm{m}$ particles respectively. 
Each particle size bin represents particles with similar settling velocities. 
The settling velocity affects modeled transport in
two main ways. Gravitational settling combined with wind
shear can cause different size particles to be transported in different directions and at different speeds. 
Deposition will decrease the amount of material in the transported plume, with larger particles remaining closer to the ground and losing more mass.

In our setup, we can compare transport of the different particle sizes by applying a constant emissions vector and comparing concentrations (in arbitrary units) due to each
particle size. 

\subsection{Transport differences}

Differences in modeled concentrations can be due to either differences in modeled emissions or modeled transport. 
In Run 10 particles were released only from the center of each source point, while in Run 11 particles were released 

\subsection{Drag partition correction}

Modeling emissions using equation~\ref{eq:westphal} and modeling transport with HYSPLIT driven by ERA5 produced quite good predictions. These are shown in Figure~\ref{fig:run1}.
For comparision, prediction using constant emissions are shown as well. Clearly, modeling emissions correctly is important.

However, when HYSPLIT was driven by the 27km WRF, model predictions were overestimated significantly. 
Figure~\ref{fig:wind_ustar} shows time series of 10 meter winds and friction velocities for all the source points for both the ERA5
and the 27km WRF. Friction velocities output by WRF are higher and emissions are quite sensitive to these values. 

The overestimation is even more extreme for the 9km WRF  and 3km WRF.
The 9km WRF runs output friction velocities up to 160 cm/s, more than twice the value of the highest .

The friction velocity provided by the NWP model represents the total shear stress on the surface and generally areas with higher aerodynamic roughness will 



~\cite{Darmenova} suggests calculating $u_*$ from ten meter wind speeds using Equation~\ref{eq:lawofwall} and an aeolian roughness length which may be much smaller than aerodynamic roughness length from the NWP model.  Doing so is similar to applying the 
drag partition correction described in~\citep{Marticorena97,MacKinnon04,Darmenova09}. 
In a neutral PBL where the velocity profile is logarithmic, the relcalulated friction velocity, $u_*'$, will be related to the model friction velocity $u_*$ in the following way

$$\frac{u_*'}{u_*} = 1 - \frac{\mathrm{ln}\frac{z_o}{z_{os}}}{\mathrm{ln}\frac{10\mathrm{m}}{z_{os}}} $$

%~\cite{Marticorena97} assumes an internal boundary layer, IBL, develops between roughness elements and there is a height, h, at which the logarithmic velocity profile of the IBL transitions to the logarithmic velocity profile of the atmospheric boundary layer. 

Following this procedure using  $z_0=0.001m$  resulted in model predictions which were comparable to those produced with the ERA5 data. 

Figure~\ref{fig:ustar27} plots friction velocity output by the model vs. friction velocity calculated from ten meter wind speeds. Figure~\ref{fig:run11} shows model predictions and observations at the five measurement sites.

However, using this procedure on the modeled emissions with the ERA5 data resulted in much worse predictions.

For the 9km WRF, using $z_0=0.0001m$ yielded  comparable results.

It is worth noting that in ~\cite{Darmenova09} and elsewhere, the drag partition correction is applied to the threshold friction velocity and the vertical mass flux is.


\subsection{Sensitivity Analysis}

Here we investigate the relationship between the emissions and the friction velocity, $u_*$, provided by the meteorological model.
To begin with, it is assumed that the mass flux is related to the friction velocity in the following manner.

\begin{equation}
\Phi(u_*, u_{*t}) =\begin{cases}
Au_*^N \;\;\; u_{*} \geq u_{*t} \\
0 \;\;\;\;\;\;  u_* < u_{*t} 
\end{cases}
\label{eq:massflux}
\end{equation}

Where A, N, and $u_{*t}$ are unknown. 
%Other unknowns not investigated here include the particle size distribution, source area extent, depenedence on precipitation.

We assume that $A$ can be any
positive real number. $N$ and $u_{*t}$ can take on any of the following values, in any combination.
$$ u_{*t} = 10,15,20,25,30,35,40,45,50 \; \mathrm{cm} \; \mathrm{s}^{-1}$$
$$ N = 1,2,3,4,5,6,7,8,9 $$

We also look at the following relationships

\begin{equation}
\Phi(u_*, u_{*t}) =\begin{cases}
A(u_*^3 - u_{*t}^3)  \;\;\; u_{*} \geq u_{*t} \\
0 \;\;\;\;\;\;  u_* < u_{*t} 
\end{cases}
\label{eq:leadbetter}
\end{equation}

%This same as next one.
%\begin{equation}
%\Phi(u_*, u_{*t}) =\begin{cases}
%Au_* (u_*^2 - u_{*t}^2)  \;\;\; u_{*} \geq u_{*t} \\
%0 \;\;\;\;\;\;  u_* < u_{*t} 
%\end{cases}
%\label{eq:marticorena}
%\end{equation}

\begin{equation}
\Phi(u_*, u_{*t}) =\begin{cases}
Au_*^3 (1-\frac{u_{*t}^2}{u_{*}^2})  \;\;\; u_{*} \geq u_{*t} \\
0 \;\;\;\;\;\;  u_* < u_{*t} 
\end{cases}
\label{eq:shao}
\end{equation}

\begin{equation}
\Phi(u_*, u_{*t}) =\begin{cases}
Au_*^4 (1-\frac{u_{*t}}{u_{*}})  \;\;\; u_{*} \geq u_{*t} \\
0 \;\;\;\;\;\;  u_* < u_{*t} 
\end{cases}
\label{eq:gillette}
\end{equation}

These functional forms have been suggested in ~\cite{Shao08}, ~\cite{Gillette}, ~\cite{Marticorena}, ~\cite{Leadbetter12}.

The constant $A$ is estimated by plotting observations vs. forecasts and performing a linear regression to find the slope.

????
To reduce bias, cumulative distribution function, CDF,  matching is performed. CDF matching can help account for
background concentrations.

Then the Pearson correlation coefficient, $r$,  and root mean square error, rmse, are calculated 


