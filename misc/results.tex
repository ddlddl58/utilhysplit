
\section{Results}

%-------------------------------------------
Modeling emissions using equation~\ref{eq:westphal} and modeling transport with HYSPLIT driven by ERA5 produced quite good predictions. These are shown in Figure~\ref{fig:run1}.
For comparision, prediction using constant emissions are shown as well. Clearly, modeling emissions correctly is important.

However, when HYSPLIT was driven by the 27km WRF, model predictions were overestimated significantly. 
Figure~\ref{fig:wind_ustar} shows time series of 10 meter winds and friction velocities for all the source points for both the ERA5
and the 27km WRF. Friction velocities output by WRF are higher and emissions are quite sensitive to these values. 

The overestimation is even more extreme for the 9km WRF  and 3km WRF.
The 9km WRF runs output friction velocities up to 160 cm/s, more than twice the value of the highest .

%-------------------------------------------

\subsection{Importance of transport}

Predicting high ash concentrations at locations farther away from the source may be easier than predicting high ash concentraions for more proximal locations. For instance, the :w
:q

It bears mentioning that
modeling transport correctly was more important for stations which were further away from the source, while modeling emissions
correctly was more important for stations close to the source.
Also, the differences between w27A and w27B were smaller at stations further away (Hval and Gren). 

The friction velocity provided by the NWP model represents the total shear stress on the surface and generally areas with higher aerodynamic roughness will 


Figure~\ref{fig:correlationsa} shows the correlation coefficient as a function of functional form, $N$, $u_{*t}$ for the period
of 5/25/2010 through 6/30/2010 using PM$_{10}$ measurements at the four stations shown in Figure~\ref{fig:locations}.
In the plots, $N=-1, -2, -3, -4$ corresponds to the use of Equations~\ref{eq:leadbetter}, ~\ref{eq:marticorena}, ~\ref{eq:shao}, ~\ref{gillette} respectively.

Figure~\ref{fig:rmse}

%Figure~\ref{fig:correlationsc} shows the correlation coefficient as a function of $N$ and $u_{*t}$ for the period
%of 10/01/2010 through 2/15/2011 using measurements from the OPC at Drangshilidardalur.

For the $PM_{10}$ meausurements, sources which experienced friction velocities greater than $45 \mathrm{cm} \; \mathrm{s}^{-1}$
did not place ash at any of the meausurement sites, thus placing the friction velocity above 50~cm~s$^{-1}$ resulted in zero
correlation.  
The more likely releationships tend to  pair higher values of $N$ with lower values of $u_{*t}$. This indicates that while
contributions from sources with low values of friction velocity are important, they must release much less mass than sources
which experience high values of $u_{*}$.

For values of $N$ lower than 4 or 5, values of 30 or 35 for $u_{*t}$ seem likely to result in higher correlations 
 which is consistent with what
has been seen in the literature ~\citep{Leadbetter12, Folch14}.

The most likely relationships found with the $PM_{10}$ measurements from the period of 5/25/201 to 6/30/2010 have
a high overlap with the most likely relationships found with the OPC counter from 10/1/2011 to 2/15/2011.
This demonstrates that assimilating measurements in this manner has potential for improving resuspended ash forecasts.

Modeled concentrations are compared to measurements at each of the stations in Figures~\ref{fig:concplota} through ~\ref{fig:concplotskogar}.
Three modeled concentrations which utilize different values of $N$ and $u_{*t}$ are shown in each plot.
For the PM$_{10}$ measurements, $N=8$ and $u_{*t}=25$ was the most likely, whie for the OPC measurements, $N=7$ and $u_{*t}=10$
was the most likely.  As can be seen, the correlations with the OPC measurements were much lower than the correlations with the PM10 meausurements.
Several high concentration events which were measured by the OPC were not captured by the model and the duration of modeled events tended to be
too short. It may be that some sources were missing for this. This was the only station located within the resuspended source area. It could
be the case that localized conditions not captured by the NWP model caused some resuspension events.

Figure~\ref{fig:concplota} shows concentrations as a function of time at the Heimaland station.
Figure~\ref{fig:concplotb} shows concentrations as a function of time at the Hvolsvollur station.
Figure~\ref{fig:concplotc} shows concentrations as a function of time at the Hvaleyrarholt station.
Figure~\ref{fig:concplotd} shows concentrations as a function of time at the Grensavegur station.
Figure~\ref{fig:concplotskogar} shows concentrations as a function of time at the Drangshilidardalur station.


\conclusions  %% \conclusions[modified heading if necessary]

The results agree with the invesgitation by ~\cite{Folch14} which predicted resuspended ash concentrations in South America using functional
relationships Equation~\ref{eq:shao}, ~\ref{eq:marticorena}, and ~\ref{eq:massflux} with $N=4$ and $u_{*t}=30\mathrm{cm}\;\mathrm{s}^{-1}$.
~\cite{Folch14} found that the simplest emission scheme 

%Transport and dispersion models aimed at forecasting resuspension of volcanic ash should consider using
%relatively low cutoff thresholds for friction velocity, of only 10 to 25 cm~s$^-1$ paired with a mass flux
%relationship which specifies a steep increase in mass flux with $u_*$. 

%Since deposit properties may change over time and from area to area, 
%a Baysian inference scheme such as demonstrated here
%can provide information on the most effective mass flux relationships to use for the area and model. 
%Here we show that data collected over a month period at four measurement stations improved forecasts for a subsequent 4 month time period
%at a different location.
(more work in the results section to show this is the case but expect it to be so as liklihoods overlapped significantly).

This scheme may also be useful for applying to dust storms.




