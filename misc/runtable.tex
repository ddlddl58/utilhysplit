
Particle diameters of 1, 5, 10, and 20 $\mu\mathrm{m}$ were modeled. All particles were given a shape factor of 1.
The 1,5 and 5$\mu\mathrm{m}$ particles were given a density $2.5\mathrm{g}\;\mathrm{cc}^{-1}$ while the 20$\mu\mathrm{m}$ particles had a density of 
$2.2\mathrm{g}\;\mathrm{cc}^{-1}$. In HYSPLIT, the particle diameter, density and shape are used
to calculate a settling velocity. The ~\cite{Ganser93, dare} formulation was used to calculate settling velocity.
The output concentration grid for each run has resolution 50 meters in he vertical. Horizontal resolution for a run set was  either $0.2 \times 0.2$ or $0.05 \times 0.05$ degrees.
Table~\ref{tab:runs} associates a run set identifier with pertinent model inputs.
Model inputs not specified in the table remained the same for all run sets.


 \begin{table}
 \caption{Summary of sets of HYSPLIT runs.}
 \centering
 \begin{tabular}{l c c c}
 \hline
  Run Set  & NWP model & Concentration  & Number release points  \\
   & NWP model & Grid Resoluton & per source \\
 \hline
   E     & ERA5    & $0.2^o$  & 1  \\
   W27A  & WRF 27km& $0.2^o$  & 1  \\
   W27B  & WRF 27km& $0.05^o$ & 81 \\
   W9    & WRF 9km & $0.2^o$  & 1  \\
   W3    & WRF 3km & $0.05^o$ & 1  \\
 \hline
 %\multicolumn{2}{l}{$^{a}$Footnote text here.}
 \label{tab:runs}
 \end{tabular}
 \end{table}

\begin{table}
 \caption{Postprocessing}
 \centering
 \begin{tabular}{l c c c}
 \hline
  Identifer & Emission & Source set  & particle size  \\
 \hline
r1s4psd & WE & s4 & PSD \\ 
r1s4p5 & WE & s4 & 5um  \\

 \hline
 %\multicolumn{2}{l}{$^{a}$Footnote text here.}
 \label{tab:results}
 \end{tabular}
 \end{table}

